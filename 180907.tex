\documentclass[12pt]{article} % 12 -- размер шрифта

\usepackage{cmap} % Чтобы можно было копировать русский текст из pdf
\usepackage[T2A]{fontenc}
\usepackage[russian,english]{babel}
\usepackage[utf8]{inputenc} % Проверьте, что кодировка файла -- тоже utf8
\usepackage{amsmath, amssymb} % Чтобы юзать математические символы

\usepackage{hologo} % Логотип LaTex
\usepackage[russian]{hyperref} % http ссылки на внешние источники
\begin{document}
	\begin{center}
		\textbf{Задание 1}\\
	\end{center}
	\textbf{Дано:} гиперкуб.\\
	\\
	\textbf{Алгоритм:}
	\begin{enumerate}
		\item[1.] Для начала заматим, что граф $G$, вершинами которого являются вершины гиперкуба, а ребрами - ребра гиперкуба, двудольный.
		\item[2.] В полученном графе $G$ удалим все вершины, покрашенные в черный цвет (граф ($G_1$), очевидно, остался двудольным).
		\item[3.] Найдем в графе ($G_1$) максимальное независимое множество. Перекрасим вершины не входящие в это множество.
	\end{enumerate}
	Время: О(VE)\\
	\\
	Корректность: допустим, что можно оставить непрекрашенными больше вершин. Заметим, что какие-то две из них гарантированно соединены ребром,
	иначе возникнет противоречие с максимальностью независимого множества. Противоречие.\\
	\\
	\begin{center}
		\textbf{Задание 2}\\
	\end{center}
	\textbf{Дано:} граф G.\\
	\\
	\textbf{Алгоритм:}\\
	\begin{enumerate}
		\item[1.] Найдем какое-нибудь минимальное вершинное покрытие графа G. Пусть его размер sz.
		\item[2.] Удалим из графа $G$ лексиграфически минимальную вершину и все инцидентные ей ребра. Получим граф $G_1$.
		\item[3.] Найдем какое-нибудь минимальное вершинное покрытие в графе $G_1$. Его размер $sz_1$
		\begin{enumerate}
			\item[$sz_1 == sz$] Возвращаемся к пункту 2 и берем вторую лексиграфически минимальную вершину.\\
			\item[$sz_1 == sz - 1$] Возвращаемся к пункту 2, но теперь $G := G_1$, $sz := sz_1$.
		\end{enumerate}
	\end{enumerate}
	Время: $O(V^2E)$ - для каждой вершины не более одного поиска вершинного покрытия.\\
	\\
	Корректность: Перебор вершин идет в лексиграфическом порядке -> Очевидно.\\
	\\
	\begin{center}
		\textbf{Задание 3}\\
	\end{center}
	\textbf{Дано:} граф G.\\
	\\
	\textbf{Алгоритм:} \\
	\begin{enumerate}
		\item[1.] Разделим каждую вершину на две, получим двудольный граф $G_1$. В первой доле вершины, отвечающие за исходящие, во второй за входящие ребра.
		\item[2.] Найдем максимальное паросочетание. Если оно не совершенно, то разбить вершины на циклы нельзя,
				   иначе просто восстановливаем циклы, переходя по ребрам из паросочетания.
	\end{enumerate}
	Время: O(VE)\\
	\\
	Корректность: если совершенное паросочетание нашлось, то восстановить разделение на циклы довольно легко - просто 
				  берем вершину, ходим по ребрам из $G_1$, причем из второй доли в первую ведут только ребра из паросочетания, если не нашли, то допустим, что разбиение на циклы есть, но такое разбиение дает нам 
				  и совершенное паросочетание (просто ребра этих циклов). Противоречие.\\
	\\
	Так как мы только что установили биекцию между наборами циклами и паросочетаниями, то задача поиска минимального
	набора циклов сводится к задаче поиска минимального совершенного паросочетания, а это мы уже умеем делать.\\
	\\
	\begin{center}
		\textbf{Задание 4}\\
	\end{center}
	\textbf{Дано:} граф G.\\
	\\
	\textbf{Алгоритм:}\\
	\begin{enumerate}
		\item[1.] Разделим каждую вершину на две, получим двудольный граф $G_1$. В первой доле вершины, отвечающие за исходящие, во второй за входящие ребра.
		\item[2.] Найдем максимальное паросочетание. Множество ребер из паросочетания и будет ответом.
	\end{enumerate}
	Время: O(VE)\\
	\\
	Корректность: степень каждой вершины $\le 2 \Rightarrow$ граф, составленный из ребер из мультимножества представляет 
				  объединение циклов, путей и изолированных вершин (только если и в исходном графе эта вершина была изолирована). Для максимизации размера мультимножество необходимо максимизировать количество циклов. Пусть k - количество вершин не принадлежащих циклам, а m - размер паросочетания, n - количество вершин в графе. Тогда $k = n - m$. Т.к. вершина без пары - это начало/конец пути. Опять биекция между количеством циклов и размером паросочетания.\\
	\\
	\begin{center}
		\textbf{Задание 5}\\
	\end{center}
	\textbf{Дано:} набор прямых.\\
	\\
	\textbf{Алгоритм:}\\
	\begin{enumerate}
		\item[1.] Пусть прямые - вершины графа $G$. Между вершинами $i, j$ проведем ребро, если\\
		 $(line_i \parallel line_j) | ((0, line_i(0)) \in line_j)$
		\item[2.] Найдем максимальное независимое множество.
	\end{enumerate}
	Время: $O(n^3)$\\
	\\
	Корректность: очевидно\\
	\\
	\begin{center}
		\textbf{Задание 6}\\
	\end{center}
	\textbf{Дано:} граф G.\\
	\\
	\textbf{Алгоритм:}\\
	\begin{enumerate}
		\item[1.] Строим эйлеров обход графа $G$.
		\item[2.] Выкидываем из этого цикла все ребра с четными номерами вхождения (нумерация начинается с любой 	
			      вершины).
		\item[3.] Из оставшихся ребер получился $2^{k - 1}$-регулярный граф. Возвращаемся к пункту 1.
		\item[4.] $k = 1$ - осталось паросочетание.  
	\end{enumerate}
	Время: O(E). На первой итерации $O(E)$, на второй - $O(\frac{E}{2})$ и т.д.\\
	\\
	Корректность: очевидно.\\
	\\
	\begin{center}
		\textbf{Задание д2}\\
	\end{center}
	\textbf{Дано:} граф G.\\
	\\
	\textbf{Алгоритм:}\\
	\begin{enumerate}
		\item[1.] Сортируем вершины первой доли по весу.
		\item[2.] Запускаем Куна, который перебирает вершины в отсортированном порядке.
	\end{enumerate}
	Время: O(VE)\\
	\\
	Корректность: пусть есть такая вершина v, которую возьмет алгоритм, но которой нет в оптимальном ответа.\\
				  Следовательно, ни одно ребро из инцидентных ей не лежит в ответе, но, так как алгоритм хотел ее взять,
				  на расстоянии 2 от нее есть вершина с меньшим весом, возьмем первое ребро этого пути и уберем второе.
				  Мы увеличили вес паросочетания.\\
	\\
	\begin{center}
		\textbf{Задание д3}\\
	\end{center}
	\textbf{Дано:} граф G.\\
	\textbf{Алгоритм:}\\
	\begin{enumerate}
		\item[1.] Сортируем вершины первой доли по весу.
		\item[2.] Сортируем ребра каждой вершины (сначала те, которые ведут в вершины с большим весом).
		\item[3.] Запускаем Куна.
	\end{enumerate}
	\begin{center}
		\textbf{Задание д4}\\
	\end{center}
	\textbf{Дано:} граф G, его матрица смежности A (смежность вершин первой и второй доли).\\
	\textbf{Алгоритм:}\\
	\begin{enumerate}
		\item[1.] Посчитаем определитель матрицы A mod 2. Если он 0, то кол-во совершенных паросочетаний четно, иначе нечетно.
	\end{enumerate}
	Время: $O(V^3)$\\
	\\
	Корректность: совершенное паросочетание - перестановка вершин первой доли, причем $a_{\sigma(i)i} = 1$, для любого i. Тогда количество совершенных паросочетаний: \\
	\begin{center}
		$\sum_{\sigma}{\prod_{i = 0}^{n - 1}{a_{\sigma(i)i}}}$\\
	\end{center}
	Заметим, что это то же самое что и дискриминант с точность до $(-1)^{sign(\sigma)}$ в каждом слагаемом, но заметим, что $-1 \% 2 = 1 \% 2$, следовательно, можно заменить  $(-1)^{sign(\sigma)}$ на 1. Получили, что по модулю 2 дискриминант и количество совершенных паросочетаний равны.\\
\end{document}